\documentclass[arbeit=studie,oneside,BCOR=12mm]{ArbeitRST}
% Die Option BCOR legt den Rand für die Bindekorrektur links fest
% (verschiebt das ganze Dokument nach rechts auf dem Papier, damit
% Platz zum Binden ist

% bib-Datei mit den Literaturangaben
% Nutzung von biber/biblatex, Erläuterungen
% siehe Text!
% =========================================
\addbibresource{SA_Rockstroh_Literatur.bib}

% Zwei Parameter zum Verändern des Layouts
% ========================================
% \parindent -> Legt fest, mit welcher Einrückung jeder neue
%               Absatz beginnen soll
% \parskip -> Legt fest, wieviel vertikaler Abstand zwischen zwei
%             Absätzen liegen soll
%
% Tipp: Entweder parindent auf Null und parskip auf einen Wert
% ungleich Null (z.B. 2ex) oder umgekehrt. Beide Werte ungleich
% Null macht satztechnisch keinen Sinn. 1ex = Breite des 
% Buchstabens x
\setlength{\parindent}{0ex}
\setlength{\parskip}{2ex}


% Einige Einstellungen für das hyperref-Paket
% =========================================== 
% Hiermit können Links, Gleichungsnummern etc. farbig dargestellt
% werden was die Navigation im elektronischen Dokument vereinfacht. An
% dieser Stelle können Sie die Farbgebung anpassen. Druckversion bitte
% ohne farbige Links erstellen, siehe Option unten!
\hypersetup{
    unicode=false,          % non-Latin characters in Acrobat’s bookmarks
    pdftoolbar=true,        % show Acrobat’s toolbar?
    pdfmenubar=true,        % show Acrobat’s menu?
    pdffitwindow=false,     % window fit to page when opened
    pdfstartview={FitH},    % fits the width of the page to the window
    pdftitle={RST Vorlage}, % title
    pdfauthor={Author},     % author
    pdfsubject={Subject},   % subject of the document
    pdfcreator={Creator},   % creator of the document
    pdfproducer={Producer}, % producer of the document
    pdfkeywords={keyword1} {key2} {key3}, % list of keywords
    pdfnewwindow=true,      % links in new window
    colorlinks=true,        % false: boxed links; true: colored links
    linkcolor=blue,         % color of internal links (change box color with linkbordercolor)
    citecolor=green,        % color of links to bibliography
    filecolor=magenta,      % color of file links
    urlcolor=cyan           % color of external links
}

% Entfernt die farbigen Markierungen - bitte Druckversion mit dieser Option kompilieren
%\hypersetup{hidelinks}



% =================================================================
\begin{document}

% Titelseite
% ==========

% Name des Verfassers
\author{Jonathan Rockstroh}

% Geburtsort
\geburtsort{Pirna}

% Geburtsdatum
\geburtsdatum{14. Mai 1997}

% Titel der Arbeit
\title{Provisorischer Titel: Semantische Katalogisierung Regelungstechnischer Systeme}

% Untertitel
%\subtitle{}

% Angabe der Betreuer
\betreuer{Carsten Knoll}

% Datum der Einreichung
%\date{2. Februar 2222}


% Zunächst für das Vorgeplänkel römische Seitenzahlen und einfacher Seitenstil
% ============================================================================
\pagenumbering{Roman}
\pagestyle{plain}


% Titelseite erstellen
\maketitle


% Selbstständigkeitserklärung
% ===========================

% Ort der Selbstständigkeitserklärung (Standard: Dresden)
\selbstort{Pirna}

% Datum der Selbstständigkeitserklärung (Standard: aktuelles Systemdatum)
\selbstdatum{1. September 2021}

% Selbstständigkeitserklärung erstellen
\selbststaendigkeitserklaerung


% Kurzfassung / Abstract
% ======================
\kurzfassung{An dieser Stelle fügen Sie bitte eine deutsche Kurzfassung ein.}{Please insert the English abstract here.}


% Inhaltsverzeichnis
% ==================
\tableofcontents

% Ggf. Symbolverzeichnis
% ======================
\chapter*{Verzeichnis der Abkürzungen \markboth{VERZEICHNIS DER ABKÜRZUNGEN}{}} \label{ch:Symbolverzeichnis}
\addcontentsline{toc}{chapter}{Verzeichnis der Abkürzungen}
\begin{tabular}{ll}
	YAML & Yet Another Markup Language \\
	OCSE & Ontology of Control System Engineering \\
	ACKRep & Automation and Control Knowledge Repository \\
	KS & Klassifikationssystem \\
	IB & Informationsblock \\
	DGL & Differentialgleichung \\
	DAE & Differential Algebraic Equation (dt. Differential-Algebraische Gleichung) \\
	PDE & Partial Differential Equation (dt. Partielle Differentialgleichung)
\end{tabular}

% Ggf. Abbildungsverzeichnis
% ==========================
\listoffigures


% Ggf. Tabellenverzeichnis
% ========================
\listoftables


% ========================
% Beginn Inhalt der Arbeit
% ========================

% Ab hier arabische Seitenzählung und heading Seitenstil
\pagestyle{scrheadings}
\pagenumbering{arabic}

% Kapitel 1: Motivation
\chapter{Motivation}
Das Schreiben einer Seminararbeit ist in einem Elektrotechnik Studium erforderlich. Mit Hilfe von Herrn Winkler und Herrn Knoll gelangt ich an eine interessante Aufgabenstellung.  

% Kapitel 2: Aufgabenstellung
\chapter{Aufgabenstellung}
Im Rahmen dieser Studienarbeit soll eine Katalog für regelungstechnische Systeme entworfen werden. Darin sollen Modelle als Textrepräsentation und (optional) zusätzlich als implementierter Code enthalten sein. Für beide Repräsentationsarten soll es eine einheitliche Repräsentationsweise geben. Die Umsetzung so erfolgen, das neue Modelle möglichst einfach hinzugefügt werden können. Ebenso soll ein Klassifikationssystem erstellt werden mit dem die Modelle innerhalb der regelungstechnischen Theorie eingeordnet werden können. Das Klassifikationssystem soll auf eine signifikante Anzahl regelungstechnischer Veröffentlichungen angewandt werden. Außerdem sollen ausgewählte  Modelle implementiert werden.

Originale Themenstellung:
Semantische Katalogisierung und formale Repräsentation regelungstechnischer Problemstellungen, Lösungsmethoden und Systemmodelle.

In der regelungstechnischen Literatur werden eine Vielzahl von Aufgaben- und Problemstellungen betrachtet (z.B. Arbeitspunktstabilisierung, Beobachterentwurf, Parameteridentifikation, Trajektorienplanung, ...). Diese beziehen sich auf unterschiedliche Anwendungsfelder (z.B. Verfahrenstechnik, Energietechnik, Systembiologie, abstrakte Systeme). Dazu werden verschiedene Methoden eingesetzt z.B. aus den Bereichen Algebra und Funktionentheorie (Frequenzbereich der Laplacetransformation), Differentialgeometrie (Lie-Ableitungen), Optimierung, (Modellprädiktive Regelung). Zudem werden unterschiedliche Systemklassen untersucht (z.B. rationale SISO-Übertragungsfunktionen, lineare Systeme mit Totzeit, polynomiale Systeme, mechanische Systeme) und unterschiedliche Zusatzannahmen getroffen (z.B. Eingangsbeschränkungen, Parameterunbestimmtheiten, Störeinflüsse, Güteanforderungen).

Ziel der Arbeit ist es, mittels sogenannter ontologischer Methoden ein Klassifikationsystem zu erstellen und auf eine signifikante Anzahl (z.B. 50) regelungstechnischer Veröffentlichungen anzuwenden. Zudem sollen die wichtigsten Modelle aus den Veröffentlichungen in Python implementiert und mittels einer Hierarchie semantischer Eigenschaften (z.B. "nichtlinear", "Zustandsdimension: 8", "Flachheitsstatus: nicht flach") erfasst werden.


% Kapitel 3: Aktueller Stand

% Kapitel 4: Klassifikationssystem
\chapter{Klassifikationssystem}
Das \textit{Klassifikationssystem (KS)} ist eine Übersicht von Attributen die Systemen im Rahmen der Regelungstechnik zugeordnet werden können. Da eine solche Übersicht bisher nicht im gewünschten Umfang existiert wurde diese selbst erstellt. Es lehnt stark an die in \cite{KNHE20} eingeführte OCSE an von der es sich insofern unterscheidet, das im KS nur der Teilbereich des Wissens der Regelungs- und Steuerungstheorie enthalten ist, der sich auf regelungstechnische Systeme und Modelle bezieht. Die im KS verwendeten Bezeichnungen sollen in den metadaten-Dateien der Modelle bevorzugt verwendet werden um einen einheitlichen Sprachgebrauch zu erreichen. 

\section{Aufbau}
Das KS ist ein Graph der aus Knoten und Kanten besteht. Jeder Knoten enthält ein Attribut\footnote{Knoten und Attribute werden in diesem Kapitel synonym verwendet.}. Die Kanten sind beschriftete Pfeile zwischen Knoten, welche einen Zusammenhang von zwei Attributen zeigen. Das Attribut des Kantenursprungs ist spezifischer als das Attribut des Kantenendes. Die Beschriftung der Kanten legt die genaue Art des Zusammenhanges fest. Für die Verwendung in den Metadaten-Dateien hat haben die dafür verwendbaren Knoten einen Werteintrag. Der Typ (boolean, string, list etc.) und gegebenenfalls die konkreten Werte, welche der Werteintrag annehmen kann sind im KS gegeben.

Es gibt drei Attribute die kein Kantenursprung sind. Diese stellen die Hauptkategorien des KS dar.
\textbf{Mathematische Eigenschaften}: \\
Umfasst Eigenschaften die durch die mathematische Repräsentation des Modells gegeben sind. % ... die aus der math. Rep. direkt hervor gehen.

\textbf{Systemeigenschaften}: \\
Umfasst Eigenschaften die aus der mathematischen Repräsentation mit Methoden aus der Regelungstechnik abgeleitet werden.

\textbf{Verwendung}: \\
Umfasst Anwendungsfälle und -bereiche in denen die Systeme häufig genutzt werden.

% Kapitel 5: Implementation

% ==================================
% Literaturverzeichnis
% ==================================

% Ein Literatureintrag, der nicht referenziert wird, aber im Verzeichnis erscheinen soll
\nocite{ROESSLER79}

% Literaturverzeichnis ausgeben
\printbibliography

\end{document}


%%% Local Variables:
%%% mode: latex
%%% TeX-master: t
%%% End:
