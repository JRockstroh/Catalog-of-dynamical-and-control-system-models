\chapter{Einleitung}
\label{Ch:Einleitung}
%\section{Motivation}
%\label{Ch:Einleitung:Sec:Motivation}
% Inhalt: Kurze Erklärung warum ein Katalog von Modellen sinnvoll ist und was die Idee attraktiv macht.

Die Regelungstechnik befasst sich mit der gezielten Beeinflussung und Analyse von regelungstechnischen Systemen. Dies geschieht mithilfe regelungstechnischer Systemmodelle, welche mittels Methoden der Modellbildung erstellt werden. Diese werden in der Literatur verwendet, um ausgewählte Problemstellungen (z.\,B. Regler- oder Beobachterentwurf) mittels regelungstechnischer Methoden zu lösen. Dabei ist die Kenntnis eines ausgewählten Teils der Modelleigenschaften wichtig, da bestimmte Methoden nur anwendbar sind, wenn bestimmte Modelleigenschaften (z.\,B. Steuerbarkeit oder Linearität) vorhanden sind oder diese nutzen bestimmte Modelleigenschaften, wie Flachheit oder Passivität, für die Problemlösung aus. Die erzielte Lösung wird aktuell praktisch immer mithilfe computergestützte Simulationen getestet und bewertet.

Die Systemmodelle sind also ein sehr wichtiger Bestandteil der Regelungstechnik. Die Recherche und Verwendung (z.\,B. für Analysen oder Simulationen) von Systemmodellen, erfordert aktuell jedoch oft ein nicht unerhebliches Zeitinvestment. Unter anderem, aufgrund einer uneinheitlichen (mathematischen) Darstellung und der praktisch nie vorkommenden Veröffentlichung von Quellcode. Diese Thematik wird in \autoref{Ch:Vorbetrachtung:Sec:CurrentState} näher ausführlicher betrachtet. \\
Die Kenntnis der Modelleigenschaften, ist sowohl für die Lösung regelungstechnischer Problemstellungen, als auch für die Untersuchung von Modellen auf ebendiese Eigenschaften wichtig. Ebenso ist es wichtig, dass bekannte Modelleigenschaften, zumindest innerhalb des Fachbereiches, einheitlich benannt werden, um eine, in Bezug auf die Begriffsbedeutung, eindeutige Kommunikation zu gewährleisten. 

\section{Präzisierung der Aufgabenstellung}
Um die Situation in Bezug auf die Repräsentation und Recherche von Systemmodellen zu verbessern, wird in dieser Arbeit ein Katalog regelungstechnischer Modelle entworfen. Er soll diese zentral erfassen und einheitlich repräsentiert zur Verfügung stellen. Die Modelleigenschaften sollen, einer einheitlichen Namensgebung folgend, im Katalog erfasst werden. Die einheitliche Namensgebung dieser, soll durch ein Klassifikationssystem erreicht werden, welches das Wissen bezüglich regelungstechnischer Modelle gebündelt darstellen soll. Zudem sollen möglichst viele Modelle implementiert werden, um diese für computergestützte Analysen und Simulationen verwendbar zu machen. 




%Inhalt: Aufgabenstellung in stichpunktartigen Sätzen.
%
%Im Rahmen dieser Studienarbeit soll eine Katalog für regelungstechnische Systeme entworfen werden. Darin sollen Modelle als Textrepräsentation und (optional) zusätzlich als implementierter Code enthalten sein. Für beide Repräsentationsarten soll es eine einheitliche Repräsentationsweise geben. Die Umsetzung so erfolgen, das neue Modelle möglichst einfach hinzugefügt werden können. Ebenso soll ein Klassifikationssystem erstellt werden mit dem die Modelle innerhalb der regelungstechnischen Theorie eingeordnet werden können. Das Klassifikationssystem soll auf eine signifikante Anzahl regelungstechnischer Veröffentlichungen angewandt werden. Außerdem sollen ausgewählte  Modelle implementiert werden.
%
%Original (letzter Part): Ziel der Arbeit ist es, mittels sogenannter ontologischer Methoden ein Klassifikationsystem zu erstellen und auf eine signifikante Anzahl (z.B. 50) regelungstechnischer Veröffentlichungen anzuwenden. Zudem sollen die wichtigsten Modelle aus den Veröffentlichungen in Python implementiert und mittels einer Hierarchie semantischer Eigenschaften (z.B. "nichtlinear", "Zustandsdimension: 8", "Flachheitsstatus: nicht flach") erfasst werden.

%--- Noch zu verfassen. ---