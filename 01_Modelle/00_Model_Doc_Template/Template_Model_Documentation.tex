\documentclass[10pt,a4paper]{article}
\usepackage[english]{babel}
\usepackage[utf8]{inputenc}
\usepackage{amsmath}
\usepackage{amsfonts}
\usepackage{amssymb}
\usepackage{graphicx}
\usepackage{float}

%< > ... dummy content zur Veranschaulichung

\begin{document}
	\part*{Model Documentation of the:}
	\section*{<Model Name>} % MUST - Add Model Name 
	
	%%%%%%%%%%%%%%%%%%%%%% NOMENCLATURE %%%%%%%%%%%%%%%%%%%%%%%%%%%
	
	\section{Nomenclature} % MUST
	\subsection{Nomenclature for Model Equations} % MUST
	
	%variables for model equations
	\begin{tabular}{ll}
		$<Variable>$ & <Description>		
	\end{tabular}
	
	\subsection{Nomenclature for Derivation} % SHOULD 
	
	%variables which are used additional to those in the model equations
	\begin{tabular}{ll}
		$<Variable>$ & <Description>
	\end{tabular}
	
	%%%%%%%%%%%%%%%%%%%%%% STATE + INPUT VECTOR | MDOEL EQUATIONS %%%%%%%%%%%%%%
	
	\section{Model Equations} % MUST
	
	State Vector and Input Vector:
	\begin{align*}
		\underline{x} &= (<Var1> \ <Var2>*R ...)^T \\
		\underline{u} &= (<Var3> \ <Var4>)^T
	\end{align*}

	\noindent System Equations:		
	\begin{subequations}
	\begin{align}
		\dot{<Variable>} &= <rhs> 	\\      % possible ways to write d/dt
		\frac{d}{dt} <Variable> &= <rhs> 
	\end{align}
	\end{subequations}

	%%%%%%%%%%%%%%%%%%%%%% INPUTS | PARAMETERS | OUTPUTS %%%%%%%%%%%%%%%%%%%%%%%%%%%
	\noindent
	Inputs: $<I_1,I_2,...>$ 
	\\
	Parameters: $<a,b,c,d, ...>$ % variables with constant, predefined value
	\\
	Outputs: $<O_1,...>$ % MAY
	
	%%%%%%%%%%%%%%%%%%%%%% ASSUMPTIONS %%%%%%%%%%%%%%%%%%%%%%%%%%%
	
	\subsection{Assumptions} % MAY 
		\begin{enumerate} %possible list type for the Assumptions - mögliche Formatierung für die Annahmen
			\item <Assumption>
		\end{enumerate}
	
	%%%%%%%%%%%%%%%%%%%%%% EXEMPLARY PARAMETER VALUES %%%%%%%%%%%%%%%%%%%%%%%%%%%	
	
	\subsection{Exemplary parameter values}
	\begin{tabular}{lcl} 
		Parameter & Symbol & Value \\ \hline	
	\end{tabular}

	%%%%%%%%%%%%%%%%%%%%%% DERIVATION & EXPLANATION %%%%%%%%%%%%%%%%%%%%%%%%%%%	
	
	\section{Derivation and Explanation} % SHOULD
	
	%%%%%%%%%%%%%%%%%%%%%% REFERENCES %%%%%%%%%%%%%%%%%%%%%%%%%%%
	
	\begin{thebibliography}{10}		
		\bibitem{<cite_key>}<Author1>; <Author2>: 
		\textit{<Referencetitle>}, <Place of Publication> <Year>
	\end{thebibliography}

\end{document}