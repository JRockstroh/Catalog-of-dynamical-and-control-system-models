\chapter{Einleitung}
\section{Motivation}
% Inhalt: Kurze Erklärung warum ein Katalog von Modellen sinnvoll ist und was die Idee attraktiv macht.

--- Noch zu verfassen. ---

\section{Präzisierung der Aufgabenstellung}
%Inhalt: Aufgabenstellung in stichpunktartigen Sätzen.

%Im Rahmen dieser Studienarbeit soll eine Katalog für regelungstechnische Systeme entworfen werden. Darin sollen Modelle als Textrepräsentation und (optional) zusätzlich als implementierter Code enthalten sein. Für beide Repräsentationsarten soll es eine einheitliche Repräsentationsweise geben. Die Umsetzung so erfolgen, das neue Modelle möglichst einfach hinzugefügt werden können. Ebenso soll ein Klassifikationssystem erstellt werden mit dem die Modelle innerhalb der regelungstechnischen Theorie eingeordnet werden können. Das Klassifikationssystem soll auf eine signifikante Anzahl regelungstechnischer Veröffentlichungen angewandt werden. Außerdem sollen ausgewählte  Modelle implementiert werden.

%Original (letzter Part): Ziel der Arbeit ist es, mittels sogenannter ontologischer Methoden ein Klassifikationsystem zu erstellen und auf eine signifikante Anzahl (z.B. 50) regelungstechnischer Veröffentlichungen anzuwenden. Zudem sollen die wichtigsten Modelle aus den Veröffentlichungen in Python implementiert und mittels einer Hierarchie semantischer Eigenschaften (z.B. "nichtlinear", "Zustandsdimension: 8", "Flachheitsstatus: nicht flach") erfasst werden.

--- Noch zu verfassen. ---